<<<<<<< HEAD
% Tệp: sections/05_mordau.tex

\chapter*{LỜI MỞ ĐẦU}
\addcontentsline{toc}{chapter}{Lời mở đầu}

\begin{enumerate}
    \item \textbf{Tính cấp thiết của đề tài}
    
    Trong môi trường giáo dục đại học hiện nay, nhu cầu tìm kiếm và chia sẻ tài liệu học tập, đề thi và bài giảng là vô cùng lớn. Tuy nhiên, việc chia sẻ tài liệu thường diễn ra một cách tự phát thông qua các nhóm chat, diễn đàn hoặc lưu trữ cá nhân rời rạc (Google Drive), dẫn đến tình trạng tài liệu bị trôi, khó tìm kiếm và thiếu sự kiểm chứng về chất lượng.
    
    Xuất phát từ thực tế đó, việc xây dựng một ứng dụng di động chuyên biệt giúp tập trung hóa nguồn tài liệu, hỗ trợ tìm kiếm nhanh chóng và tạo môi trường để sinh viên hỗ trợ lẫn nhau là rất cần thiết. Đề tài \textbf{"Xây dựng ứng dụng chia sẻ tài liệu StuShare trên nền tảng Android"} được lựa chọn nhằm giải quyết bài toán này, đồng thời ứng dụng các công nghệ lập trình hiện đại vào thực tiễn.

    \item \textbf{Mục đích nghiên cứu}
    
    Xây dựng hoàn thiện ứng dụng di động StuShare với các chức năng:
    \begin{itemize}
        \item Cho phép sinh viên đăng tải và chia sẻ tài liệu học tập.
        \item Cung cấp công cụ tìm kiếm, lọc tài liệu hiệu quả.
        \item Xây dựng hệ thống bảng xếp hạng để khuyến khích sự đóng góp của cộng đồng.
        \item Đảm bảo ứng dụng hoạt động mượt mà, giao diện thân thiện và bảo mật thông tin người dùng.
    \end{itemize}

    \item \textbf{Đối tượng và Phạm vi nghiên cứu}
    \begin{itemize}
        \item \textbf{Đối tượng nghiên cứu:} Quy trình chia sẻ tài liệu của sinh viên; các công nghệ phát triển ứng dụng Android hiện đại (Kotlin, Jetpack Compose, Firebase).
        \item \textbf{Phạm vi nghiên cứu:} Xây dựng ứng dụng Client trên hệ điều hành Android, tích hợp với Backend Firebase để quản lý dữ liệu và xác thực.
    \end{itemize}

    \item \textbf{Phương pháp nghiên cứu}
    \begin{itemize}
        \item \textbf{Phương pháp thu thập thông tin:} Khảo sát nhu cầu thực tế của sinh viên về việc tìm kiếm tài liệu.
        \item \textbf{Phương pháp phân tích thiết kế:} Sử dụng ngôn ngữ mô hình hóa UML để phân tích yêu cầu và thiết kế hệ thống. Áp dụng kiến trúc Clean Architecture và mô hình MVVM.
        \item \textbf{Phương pháp thực nghiệm:} Cài đặt ứng dụng bằng ngôn ngữ Kotlin trên Android Studio, kiểm thử chức năng trên máy ảo và thiết bị thật.
    \end{itemize}

    \item \textbf{Kết cấu của đồ án}
    
    Báo cáo được trình bày trong 3 chương chính:
    
    \textbf{Chương 1: Cơ sở lý thuyết và Công nghệ.} Trình bày tổng quan về lập trình di động và các công nghệ sử dụng (Kotlin, Jetpack Compose, Room, Firebase).
    
    \textbf{Chương 2: Phân tích và Thiết kế hệ thống.} Phân tích yêu cầu chức năng, thiết kế cơ sở dữ liệu, sơ đồ Use Case và thiết kế giao diện người dùng.
    
    \textbf{Chương 3: Xây dựng và Kết quả thực nghiệm.} Mô tả quá trình hiện thực hóa ứng dụng, demo giao diện và các kịch bản kiểm thử chức năng.
=======
% Tệp: sections/04_mordau.tex
\chapter*{LỜI MỞ ĐẦU}
\addcontentsline{toc}{chapter}{Lời mở đầu}

% Yêu cầu: Sử dụng danh sách có thứ tự (1, 2, 3...)
\begin{enumerate}
    \item \textbf{Tính cấp thiết của đề tài}
    
    (Đặt vấn đề, tầm quan trọng, ý nghĩa của đề tài, lý do chọn đề tài...)
    \lipsum[1] % Xóa \lipsum[1] và viết nội dung của bạn vào đây

    \item \textbf{Tình hình nghiên cứu}
    
    (Tóm tắt về những đề tài, công trình nghiên cứu đã công bố có liên quan...)
    \lipsum[2]

    \item \textbf{Mục đích nghiên cứu}
    
    (Đề tài nhằm giải quyết vấn đề gì, hướng tới kết quả gì...)
    \lipsum[3]

    \item \textbf{Nhiệm vụ nghiên cứu}
    
    (Đề tài sẽ thực hiện những nghiên cứu, những nhiệm vụ cụ thể gì...)

    \item \textbf{Phương pháp nghiên cứu}
    
    (Sử dụng phương pháp thu thập thông tin nào, phương pháp nào để nghiên cứu...)

    \item \textbf{Các kết quả đạt được của đề tài}
    
    (Tóm tắt kết quả)

    \item \textbf{Kết cấu của TTTN}
    
    (Báo cáo TTTN gồm có 3 chương: \par
    Chương 1: [Tên chương 1]. \par
    Chương 2: [Tên chương 2]. \par
    Chương 3: [Tên chương 3].)
>>>>>>> 7ef8ffeae10fc70af990ee49f3a7fc993ff46153
\end{enumerate}